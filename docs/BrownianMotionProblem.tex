\section{Brownian Motion Paths Hitting Lower Barrier}

Fix $h>0$. We will evaluate a standard Brownian motion $Y$ starting at zero at 
times $t_j=jh$, $j=1,2,\dots$. This yields the discrete time process $Y_j=Y(t_j)$ 
which is a simple random walk with IID increments distributed as $\sqrt{h}N(0,1)$.
This is a very simple Markov process.

Given a sequence of lower barrier values $d_j$ 
we say that $Y$ \textit{dies at time} $t_j$ if $Y(t_j)<d_j$.

Now we compute the probability that $Y(t)$ is still alive at time $t=t_j$. 
This is the event
$$
E_j=\big[Y(t_k)\geq d_k,\ths\forall k\leq j\big]
$$
Set $p_j:=P[E_j]$. Observing that $E_{j+1}\subseteq E_j$ and using the Markov property 
we obtain the recursion
%
\begin{align*}
p_{j+1} &= P(E_{j+1})=P(E_{j+1}\cap E_j)=P(E_j)P(E_{j+1}|E_j)\\&=
p_jP\big[Y(t_{j+1})\geq d_{j+1}\sep E_j\big]\\&=
p_jP\big[Y(t_{j+1})\geq d_{j+1}\sep Y(t_j)\geq d_j\big] = p_jH(j)
\q\text{with}\\
H(j) &:= P\big[Y(t_{j+1})\geq d_{j+1}\sep Y(t_j)\geq d_j\big]
\end{align*}
% 
Here we have used the Markov property in the following way:
%
\begin{equation}
\label{markov}
P\big(Y_{j+1}\geq d_{j+1}\ths\big|\ths Y_k\geq d_k, k=j,j-1,\dots,1\big) =
P\big(Y_{j+1}\geq d_{j+1}\ths\big|\ths Y_j\geq d_j\big).
\end{equation}
%
Sloppy formulations of the Markov property suggest as much but the Markov property 
\textit{does not imply this} and in fact in our case it is \textit{\textcolor{red}{not true}}.
This will then yield incorrect survival probabilities $p_j$ and we will verify by simulation
that the $p_j$ so computed deviate substantially from realized survival probabilities in the 
simulation.

To compute $H(j)$ above, set $t=t_j$ and $h=t_{j+1}-t_j$ so that
$t_{j+1}=t+h$ and we have
%
\begin{align*}
H(j) &=
P\big[Y(t+h)\geq d_{j+1}\sep Y(t)\geq d_j\big] \\&=
P\big[Y(t+h)-Y(t)\geq d_{j+1}-Y(t)\sep Y(t)\geq d_j\big] 
\\&=
\frac{1}{P(Y(t)\geq d_j)}
\int_{d_j}^\oo P\big[Y(t+h)-Y(t)\geq d_{j+1}-y\sep Y(t)=y\big]P_{Y(t)}(dy) 
\\&=
\frac{1}{P(Y(t)\geq d_j)}
\int_{d_j}^\oo P\big[\sqrt hN(0,1)\geq d_{j+1}-y\big]P_{Y(t)}(dy) 
\\&=
\frac{1}{F(-d_j/\sqrt{t})}
\int_{d_j}^\oo F\left(\frac{y-d_{j+1}}{\sqrt h}\right)P_{Y(t)}(dy)
\end{align*}
%
where $F(x)=P(N(0,1)\leq x)$ and we have used that the increment $Y(t+h)-Y(t)$ 
is independent of $Y(t)$ and distributed as $N(0,h)=\sqrt h N(0,1)$.
This yields the recursion
%
\begin{align}
\label{rec0}
p_{j+1} &= p_j\times H(j)\q\text{where}\\\label{rec1}
H(j) &= \frac{1}{F(-d_j/\sqrt{t_j})}
\int_{d_j}^\oo F\left(\frac{y-d_{j+1}}{\sqrt h}\right)P_{Y(t_j)}(dy),
\end{align}
%
with starting condition (note $t_1=h$)
%
\begin{equation}
\label{rec2}
p_1=P[Y(t_1)\geq d_1]=F(-d_1/\sqrt h).
\end{equation}
%
$H(j)$ as in (\ref{rec1}) is implemented as \code{fcn\_H}, see file
\code{R/BM.R}.

Using the bivariate normal cumulative distribution function $F_C(I)$
where $C$ is the covariance matrix of the distribution and $I$ a two dimensional 
rectange, we can also compute the conditional probability $H(j)$ as

\penalty-5000
%
\begin{align} 
\notag 
H(j) &=
P\left(Y(t_{j+1})\geq d_{j+1}\sep Y(t_j)\geq d_j\right)=
\frac{F_{C_j}(I_j)}{P(Y(t_j)\geq d_j)}
\\\label{Hj1}&=
\frac{F_{C_j}(I_j)}{F(-d_j/\sqrt{t_j})},
\q\text{where}\\\notag
C_j&=Cov(Y(t_j),Y(t_{j+1}))=
\begin{pmatrix}
t_j & t_j\\
t_j & t_{j+1}
\end{pmatrix}
\q\text{ and }\\\notag
I_j &= [d_j,+\oo]\times[d_{j+1},+\oo].
\end{align}
%
This is implemented as \code{fcn\_H1} in file \code{R/BM.R}.
The R-package \code{mvtnorm} provides the required multinormal distribution
function.

We check that the functions \code{fcn\_H} and \code{fcn\_H1} yield the same 
result, see \code{test\_H\_function} in file \code{R/Tests.R}.

In the code (function \code{runSimulation} in file \code{R/BM.R})
we run a simulation of Brownian paths and monitor the probability $q_j$ of 
death at time $t_j$ (by counting death events which are much rarer). 

For $j>1$ death at time $t_j$
is the event $E_{j-1}\setminus E_j$ and since $E_j\subseteq E_{j-1}$ it follows
that
%
\begin{equation}
\label{qj}
q_j=P(E_{j-1}\setminus E_j) = p_{j-1}-p_j
\end{equation}
%
with $p_0=P(E_0)=1$ (being alive at the start of the simulation).
We observe a drastic difference between theoretical and realized death
probabilities showing clearly that (\ref{markov}) fails.

We will compute the true survival probabilities $p_j$ later.

\subsection{Simulation}

We will simulate paths $Y_j=Y(t_j)$ starting from $Y_0=0$ and keep count how
many of these paths die at each time step $t_j$. This gives us an empirical 
probability $q^e_j$ of death at time $t_j$ which we will compare with the theoretical value $q_j$ above.

There will be a deviation due to sample variation and we will report a p-value
associated with this deviation. To do this we write the empirical probability
$q^e_j$ as the sample mean
%
\begin{equation}
\label{qej}
q^e_j=(X_1+X_2+\dots+X_n)/n
\end{equation}
%
where $X_k$ is the indicator variable defined as
$$
X_k=
\begin{cases}
1\q&\text{if }Y(t)\text{dies at time $t=t_j$ in path $k$}\\
0 &\text{else}
\end{cases}
$$
Then the $X_k=0,1$ are IID with $P(X_k=1)=q_j$ from which it follows that
$E[X_k]=q_j$ and $Var(X_k)=q_j(1-q_j)$. By the central Limit Theorem the sample
mean $q^e_j$ in (\ref{qej}) is approximately normally distributed with mean
$E[q^e_j]=q_j$ and variance 
$$
Var(q^e_j)=q_j(1-q_j)/n
$$ 
and because the sample 
size $n$ in our simulation is extremely large (e.g. 4 million) this 
approximation is highly accurate.

Setting $\sigma_j:=\sqrt{Var(q^e_j)}$ the variable $(q^e_j-q_j)/\sigma_j$ is
approximately standard normal leading to a very accurate two sided 
p-value for a deviation $\delta$ of $q^e_j$ from the theoretical value $q_j$
as
%
\begin{align}
\notag
pValue(\delta)&:=
P\left(|q^e_j-q_j|\geq\delta\right)\\\notag&=
P\left(|q^e_j-q_j|/\sigma_j\geq\delta/\sigma_j\right)\\\label{pValue}&=
2*P\left(N(0,1)\leq-\delta/\sigma_j\right)=
2*F\left(-\frac{\delta\sqrt n}{\sqrt{q_j(1-q_j}}\right).
\end{align}
%
In our simulation this is extremely close to zero for $j\geq 4$ from which it 
follows that the $p_j$ as computed in (\ref{rec0}), (\ref{rec1}) above are incorrect.
This verifies that (\ref{markov}) fails.

\subsection{Correct survival probabilities}

The probability $p_j$ of survival to time $t_j$ can be obtained directly from the multivariate
normal distribution function $F_C(I)$ where $C$ is the covariance matrix of the distribution and
$I$ a rectangle in $\bbR^j$. Indeed
%
\begin{align}
\label{pj}
p_j&=P\big(Y_j\geq d_j,Y_{j-1}\geq d_{j-1},\dots,Y_1\geq d_1)=F_{C_j}(I_j),\q\text{where}
\\\notag
I_j&=[d_1,+\oo)\times[d_2,+\oo)\times\dots\times[d_j,+\oo)
\end{align}
%
and $C_j$ is the covariance matrix
%
\begin{equation}
\label{Cj}
C_j=Cov(Y_1,Y_2,\dots,Y_j)=(t_i\wedge t_k)_{i,k=1}^j.
\end{equation}
%
The multivariate distribution function $F_C(I)$ is implemented in the R-package \textit{mvtnorm}
and we will use this to compute the true survival probabilities $p_j$ from (\ref{pj}).


